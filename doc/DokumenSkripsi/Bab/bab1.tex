\chapter{Pendahuluan}
\label{chap:pendahuluan}

\section{Latar Belakang}
\label{sec:latar_belakang}
Tradisi iman Katolik mewariskan kepada kita sejumlah tokoh pejuang dan pembela nilai serta paham hidup
yang mengangkat harkat dan martabat manusia, yang disebut ``Orang-Orang Kudus''\cite{latarbelakang1}. Orang-orang kudus
terdiri dari tua-muda, rohaniawan/wati, bapak-ibu, perawan-janda, raja-rakyat jelata, cendekiawan- orang
tidak berpendidikan, yang berasal dari berbagai suku bangsa, ras dan budaya.

Pada agama Katolik, seseorang yang ingin menjadi Katolik, pertama-tama harus melewati tahapan
sakramen baptis. Pada agama Katolik, terdapat beberapa sakramen, seperti sakramen ekaristi, dan sakramen krisma (penguatan). Salah satu sakramen yang akan
dibahas pada penelitian ini adalah tahapan sakramen baptis. Sakramen baptis biasanya diikuti pada saat
masih bayi (sekitar dua-tiga bulan setelah kelahirannya). Jika orang dewasa yang baru saja ingin menjadi
Katolik, maka harus mengikuti pelajaran agama (katekumen/katekisasi) terlebih dahulu sebelum dibaptis.

Sangat dianjurkan pemilihan nama baptis melalui pertimbangan tertentu sehingga nama baptis mempunyai makna bagi orang yang bersangkutan. Berbagai pertimbangan yang dapat digunakan pada pemilihan
nama baptis yaitu perilaku atau profesi, tanggal lahir, tanggal pembaptisan, jenis kelamin, dan lain sebagainya. Pemilihan nama baptis bagi sebagian orang mungkin tidak mudah, tetapi dengan kemajuan teknologi
yang ada saat ini, maka pemilihan nama baptis akan lebih mudah untuk orang-orang yang ingin dibaptis.

Seiring dengan perkembangan teknologi, terutama pada bidang informatika, penggunaan SPK (Sistem Pendukung Keputusan) atau yang biasa disebut DSS (\textit{Decision Support System}) sangatlah penting. Penggunaan
SPK dalam pemilihan nama baptis didapatkan dengan mudah melalui berbagai metode algoritma.
SPK atau yang biasa disebut DSS adalah bagian dari sistem informasi berbasis komputer yang digunakan
untuk mendukung pengambilan keputusan dalam suatu organisasi, perusahaan, ataupun seseorang. Metode
dalam SPK ada 4 yaitu \textit{Simple Additive Weighting} (SAW), \textit{Grid Analysis} (GA), \textit{Weighted Products} (WP)
dan \textit{Multi Attribute Global Inference of Quality} (MAGIQ). Dalam pemilihan nama baptis, peran SPK sangat
dibutuhkan, karena dalam memilih nama baptis pasti banyak sekali informasi yang harus dipertimbangkan
agar dapat mendapatkan pilihan yang terbaik untuk nama baptis.

Metode SAW dipilih karena metode ini sesuai dengan kasus pemilihan nama baptis pada agama Katolik, karena pada metode ini terdapat pilihan utama dan pilihan alternatif. Pilihan utamanya dapat berupa nama dan deskripsi Santo-Santa yang benar-benar tepat untuk orang tersebut, sedangkan untuk pilihan alternatifnya dapat berupa nama dan deskripsi Santo-Santa juga, tetapi tidak benar-benar tepat. Terdapat pilihan-pilihan tersebut, agar orang yang ingin memilih nama baptis tidak terlalu terpaku pada 1 pilihan saja.

Berdasarkan penjelasan di atas, maka solusi untuk menyelesaikan masalah kesulitan dalam memilih nama
baptis adalah dengan dibuatnya penelitian ini, yaitu membuat aplikasi SPK pemilihan nama baptis Katolik.

\section{Rumusan Masalah}
\label{sec:rumusan_masalah}
Berikut adalah susunan permasalahan yang akan dibahas pada penelitian ini:
	\begin{enumerate}
		\item Pertimbangan apa saja yang dapat menentukan nama baptis pada agama Katolik?
		\item Bagaimana cara kerja algoritma \textit{Simple Additive Weighting} (SAW)?
		\item Bagaimana mengembangkan algoritma \textit{Simple Additive Weighting} (SAW)?
	\end{enumerate}
	
\section{Tujuan}
\label{sec:tujuan}
Berdasarkan rumusan masalah yang telah dibuat, maka tujuan penelitian ini dijelaskan ke dalam poin-poin sebagai berikut:
	\begin{enumerate}
		\item Mempertimbangkan cara menentukan nama baptis pada agama Katolik.
		\item Mempelajari cara kerja algoritma \textit{Simple Additive Weighting} (SAW).
		\item Mempelajari cara mengembangkan algoritma \textit{Simple Additive Weighting} (SAW).
	\end{enumerate}
	
\section{Batasan Masalah}
\label{sec:batasan_masalah}
Penelitian ini dibuat berdasarkan batasan-batasan sebagai berikut:
	\begin{enumerate}
		\item Metode SPK yang digunakan adalah metode \textit{Simple Additive Weighting} (SAW).
		\item Pemilihan nama baptis hanya dilakukan oleh orang Katolik.
		%\item Play Framework yang digunakan selama penelitian ini adalah versi 2.4.3.
		%\item \textit{Porting} Kode KIRI \textit{Dashboard Server Side} yang dilakukan adalah berdasarkan versi terbaru dari Github dengan \textit{username}: ``pascalalfadian''\cite{kiridashboard}.
	\end{enumerate}
	
\section{Metode Penelitian}
\label{sec:metode_penelitian}
Berikut adalah metode penelitian yang digunakan dalam penelitian ini:
	\begin{enumerate}
		\item Melakukan wawancara ke pastor mengenai syarat apa saja yang diperlukan untuk memilih nama baptis.
		\item Membuat kuisioner dan menyebarkannya di kalangan gereja maupun orang-orang Katolik.
		\item Melakukan studi literatur mengenai metode \textit{Simple Additive Weighting} (SAW), PHP, bootstrap, dan MySQL.
		\item Menganalisa kebutuhan berdasarkan hasil kuisioner dan wawancara.
		\item Merancang desain web SPK Pemilihan Nama Baptis Katolik menggunakan bahasa PHP.
		\item Merancang fitur - fitur yang akan dibuat dalam bahasa PHP menggunakan algoritma \textit{Simple Additive Weighting} (SAW).
		\item Melakukan pengujian terhadap fitur-fitur yang sudah dibuat.
	\end{enumerate}

\section{Sistematika Penulisan}
\label{sec:sistematika_penulisan}
Setiap bab dalam penelitian ini memiliki sistematika penulisan yang dijelaskan ke dalam poin-poin sebagai berikut:
	\begin{enumerate}
		\item Bab 1: Pendahuluan, yaitu membahas mengenai gambaran umum penelitian ini. Berisi tentang latar belakang, rumusan masalah, tujuan, batasan masalah, metode penelitian, dan sistematika penulisan.
		\item Bab 2: Dasar Teori, yaitu membahas mengenai teori-teori yang mendukung berjalannya penulisan ini. Berisi tentang seputar arti baptis, nama baptis, calon baptis, cara menentukan nama baptis, metode \textit{Simple Additive Weighting} (SAW), PHP, MySQL dan bootstrap.
		\item Bab 3: Analisis, yaitu membahas mengenai analisa masalah. Berisi tentang analisis hasil kuisioner, wawancara, analisis nama baptis menggunakan metode \textit{Simple Additive Weighting} (SAW), analisis perangkat lunak.
		\item Bab 4: Perancangan, yaitu membahas mengenai perancangan yang dilakukan sebelum melakukan tahapan implementasi. Berisi tentang perancangan fitur %CRUD KIRI \textit{Dashboard Server Side} menggunakan Play Framework, perancangan basis data, dan perancangan antarmuka KIRI \textit{Dashboard} menggunakan Play Framework.
		\item Bab 5: Implementasi dan Pengujian, yaitu membahas mengenai implementasi dan pengujian aplikasi yang telah dilakukan. Berisi tentang implementasi dan hasil pengujian aplikasi.
		\item Bab 6: Kesimpulan dan Saran, yaitu membahas hasil kesimpulan dari keseluruhan penelitian ini dan saran-saran yang dapat
diberikan untuk penelitian berikutnya. Berisi tentang kesimpulan dan saran.
	\end{enumerate}